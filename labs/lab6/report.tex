\documentclass[]{vvsu}
\usepackage{csquotes}

\vvsuyear{2025}

%%%%%%%%%%%%%%%%%%%

\usepackage{graphicx} % для изображений
\usepackage{tabularray} % для таблиц
\usepackage{siunitx} % для обозначений (процент, градус)
\usepackage{listings} % для листингов кода

% Список путей, где будут искаться изображения и файлы
\graphicspath{{images/}}

% Файл со списком источников (не используется)
% \addbibresource{./references.bib}

% Автор документа
\author{С.Е. Казюта}

% Настройка стилей для листингов кода
\input{listing_styles.tex}

%%%%%%%%%%%%%%%%%%%

\begin{document}

% Шапка
\vvsuhead{\linespread{1}\selectfont{}МИНОБРНАУКИ РОССИИ\\
\vspace{10pt}Федеральное государственное бюджетное образовательное учреждение\\
высшего образования\\
\fontsize{13}{13}\selectfont{}<<ВЛАДИВОСТОКСКИЙ ГОСУДАРСТВЕННЫЙ УНИВЕРСИТЕТ>>\\
(ФГБОУ ВО <<ВВГУ>>)\\
\vspace{10pt}\fontsize{12}{12}\selectfont{}ИНСТИТУТ ИНФОРМАЦИОННЫХ ТЕХНОЛОГИЙ И АНАЛИЗА ДАННЫХ\\
КАФЕДРА ИНФОРМАЦИОННЫХ ТЕХНОЛОГИЙ И СИСТЕМ}

% Название отчета
\title{Отчет\\по лабораторной работе №6}
\subtitle{по дисциплине\\<<Информатика и программирование>>}

% Участники работы
\member{Студент\\ гр. БИН-25-2}{С.Е. Казюта}
\member{Ассистент\\ преподавателя}{М.В. Водяницкий}

% Вывод титульника
\maketitle

% Задание
% Задание
\begin{addition}{Задание}
  Выполнить задания на Python и оформить отчет по стандартам ВВГУ.
\end{addition}

  \textit{\textbf{Задание 1.}}  

Написать функцию, которая конвертирует время из одной величины в другую.

На вход подается:

* число (величина времени)
* исходная единица измерения
* единица измерения, в которую нужно перевести

Функция должна вернуть конвертированное значение

  \textit{\textbf{Задание 2.}}  

Пользователь делает вклад в банке в размере `a` рублей сроком на `n` лет

Процент по вкладу **зависит от суммы и срока**

**Зависимость от суммы:**

* каждые 10 000 рублей увеличивают ставку на 0.3%
* но суммарное увеличение не может превышать 5%
* минимальный вклад - 30 000 рублей

**Зависимость от срока:**

* первые 3 года - 3%
* от 4 до 6 лет - 5%
* более 6 лет - 2%

Необходимо написать функцию, которая рассчитывает прибыль пользователя без учета первоначально вложенной суммы

Используется сложный процент: каждый год процент начисляется на текущую сумму вклада

На вход подаются: сумма вклада и количество лет. Результат: сумма прибыли (не весь вклад, а только заработанные проценты)

  \textit{\textbf{Задание 3.}}  

Написать функцию для вывода всех простых чисел в заданном диапазоне. Нужно учитывать некорректные данные (например, начало больше конца или диапазон без простых чисел)

На вход подаются два числа: начало и конец диапазона (включительно). На выходе - список всех простых чисел или сообщение об ошибке

(Формат вывода списка простых чисел может быть любым удобным: в строку через пробел, в несколько строк и т.п.)

  \textit{\textbf{Задание 4.}}  

Реализовать функцию сложения двух матриц

При сложении двух матриц получается новая матрица того же размера, где каждый элемент - это сумма элементов с тем же индексом из двух исходных матриц

Ограничения:

* складывать можно только матрицы одинакового размера
* размер матрицы должен быть строго больше 2 (например, 3×3, 4×4 и т.д.)
* при нарушении условий нужно вывести сообщение об ошибке

На вход подаются:

1. размер матрицы `n` (для квадратной матрицы `n × n`)
2. элементы первой матрицы (по строкам, через пробел)
3. элементы второй матрицы в таком же формате

Результат - новая матрица (в том же формате), либо сообщение об ошибке

Пример (один из возможных вариантов формата):

Вход:

```txt
2
2 5
5 3
5 2
4 1
```

Выход:

```txt
7 7
9 4
```

Пример с ошибкой (слишком маленький размер, неправильный ввод и т.п.):

```txt
1
4
5
```

Выход:

```txt
Error!
```

  \textit{\textbf{Задание 5.}}  
 
Написать функцию, которая определяет, является ли строка палиндромом

Палиндром - это строка, которая читается одинаково слева направо и справа налево (обычно без учета пробелов, регистра и знаков препинания - эти правила нужно явно задать в своей реализации)

На вход подается строка. На выходе:

* `Да`, если это палиндром
* `Нет`, если это не палиндром

% Содержание
\toc

% Глава - Выполнение работы
\section{Выполнение работы}

% Подглава - Задание 1
\subsection{Задание 1}

Задача данной программы это конвертация времени из одной величины в другую.

Алгоритм работы:
\begin{vvsu_list}
  \item Пользователь вводит время, и из какой в какую величину он хочет его конвертировать.
  \item Пользователь конвертирует время.
  \item Программа возвращает конвертированное время.
\end{vvsu_list}

\begin{vvsu_figure}{Листинг программы для задания 1}{fig:1}
  \begin{minipage}{.75\textwidth}
    \lstinputlisting[language=Python,basicstyle=\fontsize{10}{10}\linespread{1}\selectfont\ttfamily]{1.py}
  \end{minipage}
\end{vvsu_figure}

Ключевые элементы кода:
\begin{vvsu_list}
  \item ino = input().split(' ',1) - разделяет ввод пользователя на информацию с которой можно работать.
  \item nume = int(ino[0][:-1]) - выделяет число.
  \item frome = ino[0][-1]	toe = ino[1] - выделяют из какой ед. измерения в какую нужно перевести время.
  \item times = ['s','m','h'] - список ед. измерений на котором держится логика программы.
  \item diff = times.index(toe) - times.index(frome) - определяетсколько раз нужно разделить или умножить время на 60 чтобы конвертировать его время из одной величины в другую.
  \item while diff != 0: - цикл отвечающий за деление или умножение времени на 60.
  \item print(str(round(nume, 3))+toe) - выводит результат.
\end{vvsu_list}

% Подглава - Задание 2
\subsection{Задание 2}

Задача данной программы это рассчёт прибыли пользователя без учета первоначально вложенной суммы.

Алгоритм работы:
\begin{vvsu_list}
  \item Пользователь сумму и время.
  \item Программа делает сложные расчёты.
  \item Программа выводит чистую прибыль.
\end{vvsu_list}

\begin{vvsu_figure}{Листинг программы для задания 2}{fig:2}
  \begin{minipage}{.75\textwidth}
    \lstinputlisting[language=Python,basicstyle=\fontsize{10}{10}\linespread{1}\selectfont\ttfamily]{2.py}
  \end{minipage}
\end{vvsu_figure}

Ключевые элементы кода:
\begin{vvsu_list}
  \item while 1 - эта часть кода нужно чтобы при ошибке ввода программа просила пользователя ввести данные ещё раз.
  \item ino = input().split(' ',1) - разделяет ввод пользователя на информацию с которой можно работать.
  \item input-money = int(ino[0]) - определяет ко-во вложенных денег.
  \item time-left = int(ino[1]) - определяет кол-во времени.
  \item if input-money < 30000: - перезапускает цикл при неверном вводе.
  \item stavka += input-money/10000*0.003 - отвечает за расчёт процентной ставки.
  \item for time-unit in range(time-left):  - расчитывает рост прибыли учитавая процентную ставку. 
  \item print(money-input-money) - выводит результат.
\end{vvsu_list}

% Подглава - Задание 3
\subsection{Задание 3}

Задача данной программы это вывести все простые числа в заданном диапазоне.

Алгоритм работы:
\begin{vvsu_list}
  \item Пользователь вводит диапазон чисел.
  \item Программа анализирует каждое число в диапазоне.
  \item Программа выводит все простые числа в заданном диапазоне.
\end{vvsu_list}

\begin{vvsu_figure}{Листинг программы для задания 3}{fig:3}
  \begin{minipage}{.75\textwidth}
    \lstinputlisting[language=Python,basicstyle=\fontsize{10}{10}\linespread{1}\selectfont\ttfamily]{3.py}
  \end{minipage}
\end{vvsu_figure}

Ключевые элементы кода:
\begin{vvsu_list}
  \item ino = input().split(' ', 1) - пользователь вводит диапазон.
  \item if int(ino[1]) < int(ino[0]): - программа проверяет пользователя на ошибку.
  \item for num in range(int(ino[0]),int(ino[1])): - цикл выводящий каждое число в заданном диапазоне.
  \item for i in range(1,int(num**0.5)+1): - цикл проверяющий сколько раз число делится на числа до своего корня.
  \item if devs == 1: - если кол-во делителей равно 1 то число простое. 
  \item print(num, end = ' ') - выводит простое число
  \item if cnt == 0:   print('Error!') - выводит ошибку если нет ни одного простого числа.
\end{vvsu_list}

% Подглава - Задание 4
\subsection{Задание 4}

Задача данной программы это реализовать функцию сложения двух матриц.

Алгоритм работы:
\begin{vvsu_list}
  \item Пользователь вводит размер матрицы.
  \item Пользователь вводит значения первой матрицы.
  \item Пользователь вводит значения второй матрицы.
  \item Программа складывает матрицы.
  \item Программа выводит результат сложения двух матриц.
\end{vvsu_list}

\begin{vvsu_figure}{Листинг программы для задания 4}{fig:4}
  \begin{minipage}{.75\textwidth}
    \lstinputlisting[language=Python,basicstyle=\fontsize{10}{10}\linespread{1}\selectfont\ttfamily]{4.py}
  \end{minipage}
\end{vvsu_figure}

Ключевые элементы кода:
\begin{vvsu_list}
  \item def matrix-master(x, merge = False, mt1 = '', mt2 = ''): - функция отвечающая за создание и слияние матриц.
  \item if merge:   matrix-input = [int(mt1[i][mt-element]) + int(mt2[i][mt-element]) for mt-element in range(x)] -складывает элементы матрицы при merge = True.
  \item else:   matrix-input = input().split(' ') - создаёт элементы матрицы на основе ввода пользователя при merge = False.
  \item if len(matrix-input) > x:   print('Error!') - выводит сообщение об ошибке при неправельном вводе.
  \item while 1:   ino = int(input()) - принимает ввод пользователя до тех пор пока он не будет правильным.
  \item if ino < 2:   print('Error!') - выводит сообщение об ошибке при неправельном вводе.
  \item matrix-1 = matrix-master(ino) - создаёт первую матрицу.
  \item matrix-2 = matrix-master(ino) - создаёт вторую матрицу.
  \item matrix-3 = matrix-master(ino, merge = True, mt1 = matrix-1, mt2 = matrix-2) - складывает первую и втоорую матрицы.
  \item for i in matrix-3:   print(*i) - выводит результат сложения двух матриц.
\end{vvsu_list}

% Подглава - Задание 5
\subsection{Задание 5}

Задача данной программы это определить, является ли строка палиндромом.

Алгоритм работы:
\begin{vvsu_list}
  \item Пользователь вводит строку.
  \item Программа проверяет является ли введённая пользователем строка палиндромом.
  \item Программа выводит положительный или отрицательный ответ.
\end{vvsu_list}

\begin{vvsu_figure}{Листинг программы для задания 5}{fig:5}
  \begin{minipage}{.75\textwidth}
    \lstinputlisting[language=Python,basicstyle=\fontsize{10}{10}\linespread{1}\selectfont\ttfamily]{5.py}
  \end{minipage}
\end{vvsu_figure}

Ключевые элементы кода:
\begin{vvsu_list}
  \item ino = input().lower().replace(' ', '') - программа обрабатывает введённую пользователем строку.
  \item half-len = int(len(ino) / 2) - программа определяет длинну половины строки с округлением в меньшую сторону.
  \item if len(ino) '/. 2 != 0:    ino = ino[:half-len] + ino[half-len + 1:] - если строка не делится без остатка на свою поливину то программа убирает лишний символ не влиящуй на результат работы программы.
  \item if ino[:half-len] == ino[half-len:][::-1]:   print('Да') - программа выводит положительный ответ если строка является палиндромом.
  \item else:   print('Нет') - программа выводит отрицательный ответ если строка не является палиндромом.
\end{vvsu_list}

\end{document}
