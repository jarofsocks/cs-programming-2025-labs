\documentclass[]{vvsu}
\usepackage{csquotes}

\vvsuyear{2025}

%%%%%%%%%%%%%%%%%%%

\usepackage{graphicx} % для изображений
\usepackage{tabularray} % для таблиц
\usepackage{siunitx} % для обозначений (процент, градус)
\usepackage{listings} % для листингов кода

% Список путей, где будут искаться изображения и файлы
\graphicspath{{images/}}

% Файл со списком источников (не используется)
% \addbibresource{./references.bib}

% Автор документа
\author{С.Е. Казюта}

% Настройка стилей для листингов кода
\input{listing_styles.tex}

%%%%%%%%%%%%%%%%%%%

\begin{document}

% Шапка
\vvsuhead{\linespread{1}\selectfont{}МИНОБРНАУКИ РОССИИ\\
\vspace{10pt}Федеральное государственное бюджетное образовательное учреждение\\
высшего образования\\
\fontsize{13}{13}\selectfont{}<<ВЛАДИВОСТОКСКИЙ ГОСУДАРСТВЕННЫЙ УНИВЕРСИТЕТ>>\\
(ФГБОУ ВО <<ВВГУ>>)\\
\vspace{10pt}\fontsize{12}{12}\selectfont{}ИНСТИТУТ ИНФОРМАЦИОННЫХ ТЕХНОЛОГИЙ И АНАЛИЗА ДАННЫХ\\
КАФЕДРА ИНФОРМАЦИОННЫХ ТЕХНОЛОГИЙ И СИСТЕМ}

% Название отчета
\title{Отчет\\по лабораторной работе №5}
\subtitle{по дисциплине\\<<Информатика и программирование>>}

% Участники работы
\member{Студент\\ гр. БИН-25-2}{С.Е. Казюта}
\member{Ассистент\\ преподавателя}{М.В. Водяницкий}

% Вывод титульника
\maketitle

% Задание
% Задание
\begin{addition}{Задание}
  Выполнить задания на Python и оформить отчет по стандартам ВВГУ.

  \textit{\textbf{Задание 1.}}  
  Дан список из 10 различных целых чисел. Необходимо найти в нем число 3 и заменить на 30.

  \textit{\textbf{Задание 2.}}  
  Дан список из 5 целых чисел. Необходимо превратить его в список квадратов этих чисел.

  \textit{\textbf{Задание 3.}}  
  Имеется список различных целых чисел. Программа должна найти наибольшее из чисел списка и разделить его на длину списка.
  
  \textit{\textbf{Задание 4.}}  
  Имеется кортеж из нескольких произвольных элементов. Необходимо этот кортеж отсортировать. Если хотя бы один элемент не является числом, то кортеж остается неизменным.

  \textit{\textbf{Задание 5.}}  
  Имеется словарь товаров в магазине. Необходимо найти товар с минимальной и максимальной ценой.

  \textit{\textbf{Задание 6.}}  
  Имеется список произвольных элементов. Необходимо на основе этого списка создать словарь, где каждый элемент списка будет и ключом, и значением.

  \textit{\textbf{Задание 7.}}  
  Имеется словарь перевода английских слов на русский, где ключ английского слово, значение - русского. Необходимо реализовать программу которая получает на ввод русское слово и результатом выдает перевод на английский.

  \textit{\textbf{Задание 8.}}  
  Реализовать игру Камень-Ножницы-Бумага-Ящерица-Спок. Программа должна запрашивать у пользователя ввод одного из вариантов. Второй вариант случайно генерирует сама программа и возвращает победителя.
  Правила игры следующие:
  \begin{vvsu_list}
    \item Ножницы режут бумагу\%
    \item Бумага покрывает камень\%
    \item Камень давит ящерицу\%
    \item Ящерица отравляет Спока\%    
    \item Спок ломает ножницы\%    
    \item Ножницы обезглавливают ящерицу\%    
    \item Ящерица съедает бумагу\%    
    \item Бумага подставляет Спока\%    
    \item Спок испаряет камень\%    
    \item Камень разбивает ножницы\%    
  \end{vvsu_list}

  \textit{\textbf{Задание 9.}}  
  Дан список слов - например:
`["яблоко", "груша", "банан", "киви", "апельсин", "ананас"]`
Необходимо создать новый словарь, где:
  \begin{vvsu_list}
    \item Ключом будет первая буква слова\%
    \item Значением - список всех слов, начинающихся с этой буквы\%
  \end{vvsu_list}

  Пример результата:
  ```python
  ('я': ['яблоко'], 'г': ['груша'], 'б': ['банан'], 'к': ['киви'], 'а': ['апельсин', 'ананас'])
  ```


  \textit{\textbf{Задание 10.}}  
  
  Дан список кортежей, где каждый кортеж содержит имя студента и его оценки, например:
  ```python
  [("Анна", [5, 4, 5]), ("Иван", [3, 4, 4]), ("Мария", [5, 5, 5])]
  ```

  Необходимо:
1. Создать словарь, где ключ - имя студента, значение - его средняя оценка
2. Найти студента с наибольшей средней оценкой и вывести его имя и средний балл

  Пример результата:
  ```txt
  Мария имеет наивысший средний балл: 5.0
  ```

\end{addition}

% Содержание
\toc

% Глава - Выполнение работы
\section{Выполнение работы}

% Подглава - Задание 1
\subsection{Задание 1}

Задача данной программы это найти в списке число число 3 и заменить на 30.

Алгоритм работы:
\begin{vvsu_list}
  \item Программа заменяет в списке число 3 на 30 .
  \item Программа выводит новый список.
\end{vvsu_list}

\begin{vvsu_figure}{Листинг программы для задания 1}{fig:1}
  \begin{minipage}{.75\textwidth}
    \lstinputlisting[language=Python,basicstyle=\fontsize{10}{10}\linespread{1}\selectfont\ttfamily]{1.py}
  \end{minipage}
\end{vvsu_figure}

Ключевые элементы кода:
\begin{vvsu_list}
  \item listy - сам список.
  \item new-listy - мы создаём новый список 
  \item replace - метод меняет каждый элемент "3"  в новом списке на "30".
  \item print - программа выводит новый список
\end{vvsu_list}

% Подглава - Задание 2
\subsection{Задание 2}

Задача данной программы это превратить список из 5 целых чисел в список квадратов этих чисел.

Алгоритм работы:
\begin{vvsu_list}
  \item Программа заменяет в списке каждое число на его же, но во второй степени.
  \item Программа выводит новый список.
\end{vvsu_list}

\begin{vvsu_figure}{Листинг программы для задания 2}{fig:2}
  \begin{minipage}{.75\textwidth}
    \lstinputlisting[language=Python,basicstyle=\fontsize{10}{10}\linespread{1}\selectfont\ttfamily]{2.py}
  \end{minipage}
\end{vvsu_figure}

Ключевые элементы кода:
\begin{vvsu_list}
  \item listy - сам список.
  \item new-listy - мы создаём новый список 
  \item i**2 - меняет каждый элемент в новом списке на его же, но во второй степени.
  \item print - программа выводит новый список
\end{vvsu_list}

% Подглава - Задание 3
\subsection{Задание 3}

Задача данной программы это найти наибольшее из чисел списка и разделить его на длину списка.

Алгоритм работы:
\begin{vvsu_list}
  \item Программа находит наибольший элемент списка.
  \item Программа делит наибольший элемент списка на длинну списка.
  \item Вывод результата.
\end{vvsu_list}

\begin{vvsu_figure}{Листинг программы для задания 3}{fig:3}
  \begin{minipage}{.75\textwidth}
    \lstinputlisting[language=Python,basicstyle=\fontsize{10}{10}\linespread{1}\selectfont\ttfamily]{3.py}
  \end{minipage}
\end{vvsu_figure}

Ключевые элементы кода:
\begin{vvsu_list}
  \item listy - сам список.
  \item max - определяем переменную под наибольший элемент списка.
  \item for - цикл проходит по всему списку и определяет наубольший её элемент.
  \item print - программа выводит максимальный элемент списка делёнай на длинну списка.
\end{vvsu_list}

% Подглава - Задание 4
\subsection{Задание 4}

Задача данной программы это сортировка кортежа и при этом, если хотя бы один элемент не является числом, то кортеж остается неизменным.

Алгоритм работы:
\begin{vvsu_list}
  \item Программа проверяет есть ли в кортеже не числа.
  \item Сортерует кортеж если в нём нет не чисел.
  \item Выводит кортеж.
\end{vvsu_list}

\begin{vvsu_figure}{Листинг программы для задания 4}{fig:4}
  \begin{minipage}{.75\textwidth}
    \lstinputlisting[language=Python,basicstyle=\fontsize{10}{10}\linespread{1}\selectfont\ttfamily]{4.py}
  \end{minipage}
\end{vvsu_figure}

Ключевые элементы кода:
\begin{vvsu_list}
  \item corty - сам кортеж.
  \item flaggy - флажок который пригодится нам позже.
  \item for - проверяет есть ли в кортеже не число отмечая это с помощью flaggy.
  \item if else - если в коретеже нету не чисел то проводит сортировку. 
  \item print - выводит кортеж.
\end{vvsu_list}

% Подглава - Задание 5
\subsection{Задание 5}

Задача данной программы это найти товар с минимальной и максимальной ценой.

Алгоритм работы:
\begin{vvsu_list}
  \item Программа проверяет каждый элемент словаря в поисках самого маленлького элемента и самого большого.
  \item Программа выводит какой список является самым большим и самым маленким.
\end{vvsu_list}

\begin{vvsu_figure}{Листинг программы для задания 5}{fig:5}
  \begin{minipage}{.75\textwidth}
    \lstinputlisting[language=Python,basicstyle=\fontsize{10}{10}\linespread{1}\selectfont\ttfamily]{5.py}
  \end{minipage}
\end{vvsu_figure}

Ключевые элементы кода:
\begin{vvsu_list}
  \item dicty - сам словарь.
  \item mini maxi min-name max-name - стартовые значения которые нужны для запуска программы.  
  \item for - прогоняет элементы словаря через if который в свою очередь определяет является элемент минимумом и максимумом.
  \item print - выводит минимум и максимум.
\end{vvsu_list}

% Подглава - Задание 6
\subsection{Задание 6}

Задача данной программы это на основе этого списка произвольных элементов создать словарь, где каждый элемент списка будет и ключом, и значением.

Алгоритм работы:
\begin{vvsu_list}
  \item Программа создаёт а после сортирует список произвольных элементов.
  \item Программа определяет каждый элемент списка как ключом так и значением в свежесозданном словаре.
  \item Программа выводит словарь.
\end{vvsu_list}

\begin{vvsu_figure}{Листинг программы для задания 6}{fig:6}
  \begin{minipage}{.75\textwidth}
    \lstinputlisting[language=Python,basicstyle=\fontsize{10}{10}\linespread{1}\selectfont\ttfamily]{6.py}
  \end{minipage}
\end{vvsu_figure}

Ключевые элементы кода:
\begin{vvsu_list}
  \item listy - программа создаёт а после сортирует список произвольных элементов.
  \item sorted() - программа сортирует список.
  \item dicty - программа создаёт пустой словарь.
  \item for - программа определяет каждый элемент listy как ключом так и значением в dicty.
  \item print - программа выводит dicty.
\end{vvsu_list}

% Подглава - Задание 7
\subsection{Задание 7}

Задача данной программы это получать на ввод русское слово и результатом выдавать перевод на английский.

Алгоритм работы:
\begin{vvsu_list}
  \item Программа получает от пользователя русское слово.
  \item Программа находит перевод слова в базе данных.
  \item Програма ввыводит перевод.
\end{vvsu_list}

\begin{vvsu_figure}{Листинг программы для задания 7}{fig:7}
  \begin{minipage}{.75\textwidth}
    \lstinputlisting[language=Python,basicstyle=\fontsize{10}{10}\linespread{1}\selectfont\ttfamily]{7.py}
  \end{minipage}
\end{vvsu_figure}

Ключевые элементы кода:
\begin{vvsu_list}
  \item dicty - база данных в формате словаря.
  \item input - ввод пользователя.
  \item for - программа проходится по dicty в поисках перевода input с помощью if.
  \item print - программа выводит перевод input. 
\end{vvsu_list}

% Подглава - Задание 8
\subsection{Задание 8}

Задача данной программы это реализовать игру Камень-Ножницы-Бумага-Ящерица-Спок. Программа должна запрашивать у пользователя ввод одного из вариантов. Второй вариант случайно генерирует сама программа и возвращает победителя.

Алгоритм работы:
\begin{vvsu_list}
  \item Программа получает от пользователя: Камень, Ножницы, Бумага, Ящерица, Спок.
  \item Программа случайный образом выбирает случайный вариант. 
  \item Программа выводит кто победил.
\end{vvsu_list}

\begin{vvsu_figure}{Листинг программы для задания 8}{fig:8}
  \begin{minipage}{.75\textwidth}
    \lstinputlisting[language=Python,basicstyle=\fontsize{10}{10}\linespread{1}\selectfont\ttfamily]{8.py}
  \end{minipage}
\end{vvsu_figure}

Ключевые элементы кода:
\begin{vvsu_list}
  \item input() - ввод пользователя.
  \item enemy-inp randint() - случайное значение.
  \item who-beats-who - словарь с игровой логикой.
  \item if - определяет кто победил на основе информации из словаря.
  \item print() - выводит победителя(если таковой иммется).
\end{vvsu_list}

% Подглава - Задание 9
\subsection{Задание 9}

Задача данной программы это создать словарь на основе списка слов в котором ключом будет являтся первая буква слова, а значением - список всех слов, начинающихся с этой буквы.

Алгоритм работы:
\begin{vvsu_list}
  \item Программа создаёт ключ из каждой первой буквы.
  \item Программа присваевает соответствующие слова как значения.
  \item Программы выводит словарь.
\end{vvsu_list}

\begin{vvsu_figure}{Листинг программы для задания 9}{fig:9}
  \begin{minipage}{.75\textwidth}
    \lstinputlisting[language=Python,basicstyle=\fontsize{10}{10}\linespread{1}\selectfont\ttfamily]{9.py}
  \end{minipage}
\end{vvsu_figure}

Ключевые элементы кода:
\begin{vvsu_list}
  \item listy - список слов.
  \item new-listy - словарь.
  \item первый for - создаёт ключ из каждой первой буквы.
  \item двойной for - для каждого ключа, присваевает как значение слово начинающееся с буквы равной его значению.
  \item print() - выводит словарь.
\end{vvsu_list}

% Подглава - Задание 10
\subsection{Задание 10}

Задача данной программы создать словарь и в нём найти студента с наибольшей средней оценкой и вывести его имя и средний балл.

Алгоритм работы:
\begin{vvsu_list}
  \item Программа создаёт словарь на основе списка.
  \item Программа находит студента с наибольшей средней оценкой и выводит его имя и средний балл.
\end{vvsu_list}

\begin{vvsu_figure}{Листинг программы для задания 10}{fig:10}
  \begin{minipage}{.75\textwidth}
    \lstinputlisting[language=Python,basicstyle=\fontsize{10}{10}\linespread{1}\selectfont\ttfamily]{10.py}
  \end{minipage}
\end{vvsu_figure}

Пояснение работы программы:
\begin{vvsu_list}
  \item listy - список с учениками и их оценками.
  \item new-listy - пустой словарь.
  \item первый for - заполняет словарь информацией из списка.
  \item второй for - находит студента с наибольшей средней оценкой с помощью if.
  \item print() - выводит студента с наибольшей средней оценкой на экран.
\end{vvsu_list}

\end{document}
