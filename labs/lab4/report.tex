\documentclass[]{Void}

\vvsuyear{2025}

%%%%%%%%%%%%%%%%%%%

\usepackage{graphicx} % для изображений
\usepackage{tabularray} % для таблиц
\usepackage{siunitx} % для обозначений (процент, градус)
\usepackage{listings} % для листингов кода

% Список путей, где будут искаться изображения и файлы
\graphicspath{{images/}}

% Файл со списком источников (не используется)
% \addbibresource{./references.bib}

% Автор документа
\author{К.С. Студент}

% Настройка стилей для листингов кода
\input{listing_styles.tex}

%%%%%%%%%%%%%%%%%%%

\begin{document}

% Шапка
\vvsuhead{\linespread{1}\selectfont{}МИНОБРНАУКИ РОССИИ\\
\vspace{10pt}Федеральное государственное бюджетное образовательное учреждение\\
высшего образования\\
\fontsize{13}{13}\selectfont{}<<ВЛАДИВОСТОКСКИЙ ГОСУДАРСТВЕННЫЙ УНИВЕРСИТЕТ>>\\
(ФГБОУ ВО <<ВВГУ>>)\\
\vspace{10pt}\fontsize{12}{12}\selectfont{}ИНСТИТУТ ИНФОРМАЦИОННЫХ ТЕХНОЛОГИЙ И АНАЛИЗА ДАННЫХ\\
КАФЕДРА ИНФОРМАЦИОННЫХ ТЕХНОЛОГИЙ И СИСТЕМ}

% Название отчета
\title{Отчет\\по лабораторной работе №4}
\subtitle{по дисциплине\\<<Информатика и программирование>>}

% Участники работы
\member{Студент\\ гр. GGG-NN-LL}{К.С. Студент}
\member{Ассистент\\ преподавателя}{М.В. Водяницкий}

% Вывод титульника
\maketitle

% Задание
% Задание
\begin{addition}{Задание}
  Выполнить задания на Python и оформить отчет по стандартам ВВГУ.

  \textit{\textbf{Задание 1.}}  

    Написать программу, которая определяет, как будет вести себя кондиционер. Если температура в помещении 20 градусов и выше, то кондиционер выключается, если меньше - включается. Температура должна вводится пользователем с консоли.

    Пример:

    ```bash
    Введите температуру: 18  
    Кондиционер включен
    ```


  \textit{\textbf{Задание 2.}}  
  
    Год делится на четыре сезона: зима, весна, лето и осень. Написать программу, которая запрашивает у пользователя номер месяца и выводит к какому сезону этот месяц относится.

    Пример:

    ```bash
    Введите номер месяца: 4  
    Это весна
    ```


  \textit{\textbf{Задание 3.}}  

    Считается, что один год, прожитый собакой, эквивалентен семи человеческим годам. При этом зачастую не учитывается, что собаки становятся абсолютно взрослыми уже к двум годам. Таким образом, многие предпочитают каждый из первых двух лет жизни собаки приравнивать к 10.5 годам человеческой жизни, а все последующие к 4.

    Написать программу, которая будет переводить собачий возраст в человеческий. Программа должна корректно обрабатывать входные данные и выводить соответствующие сообщения об ошибках:

    - Если вводится не число
    - Если вводится число меньше 1
    - Если вводится число большее 22

    Пример:

    ```bash
    Введите возраст собаки (в годах): 5  
    Возраст собаки в человеческих годах: 33.0
    ```

    Пример:

    ```bash
    Введите возраст собаки (в годах): 0  
    Ошибка: возраст должен быть не меньше 1
    ```


  \textit{\textbf{Задание 4.}}  

    Число делиться на 6 только в случае соблюдения двух условий:

    - Последняя цифра четная
    - Сумма всех цифр делиться на 3
    
    Написать программу, которая выведет делиться ли введенное число на 6 или нет.


  \textit{\textbf{Задание 5.}}  

    Написать программу, которая будет проверять пароль на надежность. Пароль считается надежным, если его длина не менее 8 символов и если он содержит:

    - Заглавные буквы латиницы
    - Строчные буквы латиницы
    - Числа
    - Специальные знаки

    В случае, если пароль не проходит по одному из условий, необходимо сообщить пользователю каким именно условиям он не удовлетворяет.

    Пример:

    ```bash
     пароль: qwerty  
    Пароль ненадежный: отсутствуют заглавные буквы, числа и специальные символы
    ```


  \textit{\textbf{Задание 6.}}  
  
    Написать программу, которая определяет, является ли введенный пользователем год високосным. Год считается високосным, если он делится на 4, но не делится на 100, либо если он делится на 400.

    Пример:

    ```bash
    Введите год: 2024  
    2024 - високосный год
    ```

  \textit{\textbf{Задание 7.}}  

    Написать программу, которая запрашивает у пользователя три числа и выводит на экран наименьшее из них. При решении нельзя использовать встроенные функции min() и max().

    Пример:

    ```bash
    Введите три числа: 8 3 5  
    Наименьшее число: 3
    ```

  \textit{\textbf{Задание 8.}}  

    В магазине проводится акция. Акция работает по следующим правилам:

    | Сумма покупки | Скидка |
    | ------------- | ------ |
    | до 1000       | 0%     |
    | 1000–5000     | 5%     |
    | 5000–10000    | 10%    |
    | более 10000   | 15%    |

    Напишите программу, которая запрашивает сумму покупки и выводит размер скидки и итоговую сумму к оплате.

    Пример:

    ```bash
    Введите сумму покупки: 7500  
    Ваша скидка: 10%  
    К оплате: 6750.0
    ```

  \textit{\textbf{Задание 9.}}  

    Написать программу, которая определяет время суток по введенному часу (целое число от 0 до 23).

    | Время | Период |
    | ----- | ------ |
    | 0–5   | Ночь   |
    | 6–11  | Утро   |
    | 12–17 | День   |
    | 18–23 | Вечер  |

    Пример:

    ```bash
    Введите час (0–23): 20  
    Сейчас вечер
    ```

  \textit{\textbf{Задание 10.}}  
  
    Написать программу, которая определяет, является ли введенное число простым. Число называется простым, если оно больше 1 и делится только на 1 и само себя. Программа должна корректно обрабатывать некорректный ввод и выводить соответствующие сообщения об ошибках.

    Пример:

    ```bash
    Введите число: 17  
    17 - простое число
    ```

    Пример:

    ```bash
    Введите число: 12  
    12 - составное число
    ```

  \end{vvsu_list}
\end{addition}
