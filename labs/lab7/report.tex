\documentclass[]{vvsu}
\usepackage{csquotes}

\vvsuyear{2025}

%%%%%%%%%%%%%%%%%%%

\usepackage{graphicx} % для изображений
\usepackage{tabularray} % для таблиц
\usepackage{siunitx} % для обозначений (процент, градус)
\usepackage{listings} % для листингов кода

% Список путей, где будут искаться изображения и файлы
\graphicspath{{images/}}

% Файл со списком источников (не используется)
% \addbibresource{./references.bib}

% Автор документа
\author{С.Е. Казюта}

% Настройка стилей для листингов кода
\input{listing_styles.tex}

%%%%%%%%%%%%%%%%%%%

\begin{document}

% Шапка
\vvsuhead{\linespread{1}\selectfont{}МИНОБРНАУКИ РОССИИ\\
\vspace{10pt}Федеральное государственное бюджетное образовательное учреждение\\
высшего образования\\
\fontsize{13}{13}\selectfont{}<<ВЛАДИВОСТОКСКИЙ ГОСУДАРСТВЕННЫЙ УНИВЕРСИТЕТ>>\\
(ФГБОУ ВО <<ВВГУ>>)\\
\vspace{10pt}\fontsize{12}{12}\selectfont{}ИНСТИТУТ ИНФОРМАЦИОННЫХ ТЕХНОЛОГИЙ И АНАЛИЗА ДАННЫХ\\
КАФЕДРА ИНФОРМАЦИОННЫХ ТЕХНОЛОГИЙ И СИСТЕМ}

% Название отчета
\title{Отчет\\по лабораторной работе №4}
\subtitle{по дисциплине\\<<Информатика и программирование>>}

% Участники работы
\member{Студент\\ гр. БИН-25-2}{С.Е. Казюта}
\member{Ассистент\\ преподавателя}{М.В. Водяницкий}

% Вывод титульника
\maketitle

% Задание
% Задание
\begin{addition}{Задание}
  Выполнить задания на Python и оформить отчет по стандартам ВВГУ.

  \textit{\textbf{Задание 1.}}  
  Написать программу, которая определяет, как будет вести себя кондиционер. 
  Если температура в помещении 20 градусов и выше, то кондиционер выключается, если меньше - включается. 
  Температура должна вводится пользователем с консол

  \textit{\textbf{Задание 2.}}  
  Год делится на четыре сезона: зима, весна, лето и осень. 
  Написать программу, которая запрашивает у пользователя номер месяца и выводит к какому сезону этот месяц относится.

  \textit{\textbf{Задание 3.}}  
  Считается, что один год, прожитый собакой, эквивалентен семи человеческим годам. 
  При этом зачастую не учитывается, что собаки становятся абсолютно взрослыми уже к двум годам. 
  Таким образом, многие предпочитают каждый из первых двух лет жизни собаки приравнивать к 10.5 годам человеческой жизни, а все последующие к 4.
  Написать программу, которая будет переводить собачий возраст в человеческий.
  Программа должна корректно обрабатывать входные данные и выводить соответствующие
  сообщения об ошибках:
  \begin{vvsu_list}
  \item Если вводится не число
  \item Если вводится число меньше 1 
  \item Если вводится число большее 22
  \end{vvsu_list}
  
  \textit{\textbf{Задание 4.}}  
  Число делиться на 6 только в случае соблюдения двух условий:
  \begin{vvsu_list}
    \item Последняя цифра четная
    \item Сумма всех цифр делиться на 3
  \end{vvsu_list}

  Написать программу, которая выведет делиться ли введенное число на 6 или нет.

  \textit{\textbf{Задание 5.}}  
  Написать программу, которая будет проверять пароль на надежность.
  Пароль считается надежным, если его длина не менее 8 символов и если он содержит:
  \begin{vvsu_list}
   \item Заглавные буквы латиницы
   \item Строчные буквы латиницы
   \item Числа
   \item Специальные знаки
  \end{vvsu_list}
  В случае, если пароль не проходит по одному из условий, необходимо сообщить пользователю каким именно условиям он не удовлетворяет.

  \textit{\textbf{Задание 6.}}  
  Написать программу, которая определяет, является ли введенный пользователем год високосным. 
  Год считается високосным, если он делится на 4, но не делится на 100, либо если он делится на 400.

  \textit{\textbf{Задание 7.}}  
  Написать программу, которая запрашивает у пользователя три числа и выводит на экран наименьшее из них. 
  При решении нельзя использовать встроенные функции min() и max().

  \textit{\textbf{Задание 8.}}  
  В магазине проводится акция. Акция работает по следующим правилам:
  \begin{vvsu_list}
    \item Сумма < 1000 => скидка - 0\%
    \item Сумма < 5000 => скидка - 5\%
    \item Сумма < 10000 => скидка - 10\%
    \item Сумма > 10000 => скидка - 15\%    
  \end{vvsu_list}
  Напишите программу, которая запрашивает сумму покупки и выводит размер скидки и итоговую сумму к оплате.

  \textit{\textbf{Задание 9.}}  
  Написать программу, которая определяет время суток по введенному часу (целое число от 0 до 23).
  \begin{vvsu_list}
    \item С 0 до 5 часов - ночь
    \item С 6 до 11 часов - утро
    \item С 12 до 17 часов - день
    \item  С 18 до 23 часов - вечер
  \end{vvsu_list}


  \textit{\textbf{Задание 10.}}  
  Написать программу, которая определяет, является ли введенное число простым. 
  Число называется простым, если оно больше 1 и делится только на 1 и само себя. 
  Программа должна корректно обрабатывать некорректный ввод и выводить соответствующие сообщения об ошибках.
\end{addition}

% Содержание
\toc

% Глава - Выполнение работы
\section{Выполнение работы}

% Подглава - Задание 1
\subsection{Задание 1}

Задача данной программы это определить работает ли кондиционер.

Алгоритм работы:
\begin{vvsu_list}
  \item Пользователь вводит температуру.
  \item Программа выводит сообщение о состоянии работы кондиционера.
\end{vvsu_list}

\begin{vvsu_figure}{Листинг программы для задания 1}{fig:1}
  \begin{minipage}{.75\textwidth}
    \lstinputlisting[language=Python,basicstyle=\fontsize{10}{10}\linespread{1}\selectfont\ttfamily]{1.py}
  \end{minipage}
\end{vvsu_figure}

Ключевые элементы кода:
\begin{vvsu_list}
  \item input('Введите температуру: ') - это функция, которая считывает ввод от пользователя и возвращает его в виде строки.
  \item def tempo(x): - это объявление функции с именем tempo, которая принимает один параметр x.
  \item Условный оператор if x < 20: ... else: ... - проверяет, меньше ли введенная температура 20 градусов. Если да, то возвращается строка 'включён', иначе - 'выключен'.
  \item print(f'Кондиционер tempo(ino)') - это вывод на экран. Здесь используется f-строка (форматированная строка), которая вызывает функцию tempo с аргументом ino (введенная строка) и подставляет результат в строку.
\end{vvsu_list}

% Подглава - Задание 2
\subsection{Задание 2}

Задача данной программы это вывести время года в зависимости от введённого числа.

Алгоритм работы:
\begin{vvsu_list}
  \item Пользователь вводит номер месяца (1-12).
  \item Программа проверяет принадлежность месяца к сезонам.
  \item При вводе неверного числа выводится сообщение об ошибке.
\end{vvsu_list}

\begin{vvsu_figure}{Листинг программы для задания 2}{fig:2}
  \begin{minipage}{.75\textwidth}
    \lstinputlisting[language=Python,basicstyle=\fontsize{10}{10}\linespread{1}\selectfont\ttfamily]{2.py}
  \end{minipage}
\end{vvsu_figure}

Ключевые элементы кода:
\begin{vvsu_list}
  \item Список (list) mes с названиями сезонов.
  \item Функция input() для ввода номера месяца.
  \item Преобразование введенной строки в целое число с помощью int(). Индексация списка: mes[i-1] (поскольку список индексируется с 0, а месяцы обычно нумеруются с 1, вычитаем 1).
  \item Использование f-строки для форматирования вывода.
\end{vvsu_list}

% Подглава - Задание 3
\subsection{Задание 3}

Задача данной программы это конвертировать собачий возраст в человеческий.

Алгоритм работы:
\begin{vvsu_list}
  \item Пользователь вводит возраст собаки в годах.
  \item Создается функция error() для проверки корректности ввода.
  \item Бесконечный цикл while 1 продолжается до получения корректных данных. При обнаружении ошибки выводится соответствующее сообщение. При отсутствии ошибок цикл прерывается.
  \item Расчет возраста собаки в человеческих годах.
  \item Вывод результата.
\end{vvsu_list}

\begin{vvsu_figure}{Листинг программы для задания 3}{fig:3}
  \begin{minipage}{.75\textwidth}
    \lstinputlisting[language=Python,basicstyle=\fontsize{10}{10}\linespread{1}\selectfont\ttfamily]{3.py}
  \end{minipage}
\end{vvsu_figure}

Ключевые элементы кода:
\begin{vvsu_list}
  \item input('Введите возраст собаки (в годах): ') - получает строку от пользователя.
  \item impоrt strign - импортирует модуль, который содержит наборы символов, такие как strign.ascii-letters (все буквы алфавита).
  \item Проверка если первый символ a является буквой (проверка с помощью a[0] in string.ascii-letters).
  \item Проверка если преобразованное в целое число значение a меньше 1.
  \item Проверка если преобразованное в целое число значение a больше 22.
  \item while 1 - Бесконечный цикл, который продолжается до тех пор, пока не будет введено корректное значение (т.е. пока не сработает break). В цикле проверяются возвращаемые функцией error коды и выводятся соответствующие сообщения об ошибках. Если ошибок нет (когда error(ino) не возвращает 1, 2 или 3), цикл прерывается.
  \item cnt = 0 - счетчик.
  \item Цикл for i in range(1, int(ino)+1) - перебирает каждый год жизни собаки от 1 до введенного значения. Если год меньше 3 (т.е. 1 или 2), то к счетчику добавляется 10.5. иначе добавляется 4.
  \item print(f'Возраст собаки в человеческих годах: {cnt}') – вывод.
\end{vvsu_list}

% Подглава - Задание 4
\subsection{Задание 4}

Задача данной программы это проверка делимости введённого числа на 6.

Алгоритм работы:
\begin{vvsu_list}
  \item Пользователь вводит число как строку.
  \item Смотрим на последнюю цифру - должна быть 0,2,4,6,8.
  \item Складываем все цифры числа, проверяем делимость суммы на 3.
  \item Если прошли обе проверки - "Делится", иначе - "Не делится".
\end{vvsu_list}

\begin{vvsu_figure}{Листинг программы для задания 4}{fig:4}
  \begin{minipage}{.75\textwidth}
    \lstinputlisting[language=Python,basicstyle=\fontsize{10}{10}\linespread{1}\selectfont\ttfamily]{4.py}
  \end{minipage}
\end{vvsu_figure}

Ключевые элементы кода:
\begin{vvsu_list}
  \item ino = input() - получаем строку от пользователя.
  \item int(ino[-1]) ⁒ 2 == 0 - берем последний символ строки (последняя цифра), преобразуем в целое число и проверяем, четное ли оно.
  \item sum([int(i) for i in ino]) ⁒ 3 == 0 - преобразуем каждый символ строки в целое число, создаем список этих цифр, суммируем их и проверяем, делится ли сумма на 3.
  \item [int(i) for i in ino] - это генератор списка, который преобразует каждый символ строки в число.
  \item if ... and ... - если обе условия истинны, то число делится на 6.
  \item print('Не делится на "6"') - в зависимости от условия выводим сообщение о делимости на 6.
\end{vvsu_list}

% Подглава - Задание 5
\subsection{Задание 5}

Задача данной программы это проверить пароль на пригодность.

Алгоритм работы:
\begin{vvsu_list}
  \item Пользователь вводит пароль.
  \item Проверяется наличие хотя бы одной заглавной буквы.
  \item Проверяется наличие хотя бы одной строчной буквы.
  \item Проверяется наличие хотя бы одного специального символа (знака пунктуации).
  \item Если все три критерия выполнены, выводится сообщение "Пароль надежный".
\end{vvsu_list}

\begin{vvsu_figure}{Листинг программы для задания 5}{fig:5}
  \begin{minipage}{.75\textwidth}
    \lstinputlisting[language=Python,basicstyle=\fontsize{10}{10}\linespread{1}\selectfont\ttfamily]{5.py}
  \end{minipage}
\end{vvsu_figure}

Ключевые элементы кода:
\begin{vvsu_list}
  \item import string - импорт модуля string для использования предопределенных наборов символов.
  \item pswrd = input('Введите пароль: ') - запрашивает у пользователя ввод пароля.
  \item err-output = 'Пароль ненадежный: отсутствуют ' - начальная строка для сообщения об ошибке. Ее длина 31 символ (если считать пробелы и буквы).
  \item cnt = 0 - счетчик для подсчета символов определенного типа.
  \item Первый цикл for проходит по каждому символу пароля и проверяет, есть ли символ в string.ascii-uppercase. Если находит, увеличивает cnt. После цикла: если cnt остался 0, то в err-output добавляется ', Заглавные буквы'.
  \item Второй цикл for проверяет наличие цифр в пароле. Если цифры не найдены (cnt остался 0), то в err-output добавляется ', Строчные буквы'.
  \item Третий цикл for проверяет наличие специальных символов. Если не находит, то в err-output добавляется ', Специальные знаки'.
  \item if len(err-output) == 31: - если длина строки err-output не изменилась (то есть не было добавлено ни одного предупреждения), то выводится "Пароль надежный". Иначе выводится строка err-output с перечислением того, чего не хватает.
\end{vvsu_list}

% Подглава - Задание 6
\subsection{Задание 6}

Задача данной программы это проверка введённого года на то високосный он или нет.

Алгоритм работы:
\begin{vvsu_list}
  \item Получить год от пользователя.
  \item Проверить условие високосного года (год делится на 4 И при этом НЕ делится на 100 ИЛИ, год делится на 400).
  \item Вывести результат.
\end{vvsu_list}

\begin{vvsu_figure}{Листинг программы для задания 6}{fig:6}
  \begin{minipage}{.75\textwidth}
    \lstinputlisting[language=Python,basicstyle=\fontsize{10}{10}\linespread{1}\selectfont\ttfamily]{6.py}
  \end{minipage}
\end{vvsu_figure}

Ключевые элементы кода:
\begin{vvsu_list}
  \item input() возвращает строку, поэтому используем int() для преобразования в целое число.
  \item Условие проверки високосного года: yr ⁒ 4 == 0: год должен делиться на 4, not(yr ⁒ 100 == 0): но не должен делиться на 100 (исключение для столетних годов), or yr ⁒ 400 == 0: если год делится на 400, то он високосный (это перекрывает исключение для столетних).
  \item Вывод результата с использованием f-строки для подстановки значения года.
\end{vvsu_list}

% Подглава - Задание 7
\subsection{Задание 7}

Задача данной программы это нахождение наименьшего числа среди трех введенных значений.

Алгоритм работы:
\begin{vvsu_list}
  \item Получить от пользователя строку с тремя числами, разделенными пробелами.
  \item Разделить строку на список подстрок.
  \item Инициализировать переменную min очень большим числом.
  \item Для каждой подстроки в списке преобразовать подстроку в число если это число меньше текущего min, обновить min этим числом.
  \item Вывести вывод.
\end{vvsu_list}

\begin{vvsu_figure}{Листинг программы для задания 7}{fig:7}
  \begin{minipage}{.75\textwidth}
    \lstinputlisting[language=Python,basicstyle=\fontsize{10}{10}\linespread{1}\selectfont\ttfamily]{7.py}
  \end{minipage}
\end{vvsu_figure}

Ключевые элементы кода:
\begin{vvsu_list}
  \item input().split() - получает ввод и разбивает на список строк.
  \item min = 9999999999999 - Инициализация min большим числом, чтобы первое же число стало минимумом.
  \item for i in yr - цикл for для перебора элементов.
  \item if int(i) < min - реобразование строки в целое число для сравнения.
  \item print(min) – вывод.
\end{vvsu_list}

% Подглава - Задание 8
\subsection{Задание 8}

Задача данной программы это подсчёт скидки на основе цены купленного товара.

Алгоритм работы:
\begin{vvsu_list}
  \item Пользователь вводит сумму покупки.
  \item На основе суммы определяется процент скидки.
  \item Выводится информация о скидке и итоговая сумма к оплате.
\end{vvsu_list}

\begin{vvsu_figure}{Листинг программы для задания 8}{fig:8}
  \begin{minipage}{.75\textwidth}
    \lstinputlisting[language=Python,basicstyle=\fontsize{10}{10}\linespread{1}\selectfont\ttfamily]{8.py}
  \end{minipage}
\end{vvsu_figure}

Ключевые элементы кода:
\begin{vvsu_list}
  \item input() - получает строку от пользователя.
  \item int() - преобразует строку в целое число.
  \item price - хранит исходную сумму покупки.
  \item if - определяет размер скидки на основе суммы покупки: Если price < 1000, то i = 0, Если price = 1000-5000, то i = 5, Если price = 5000-10000, то i = 10, Если price = 10000, то i = 15
\end{vvsu_list}

% Подглава - Задание 9
\subsection{Задание 9}

Задача данной программы это определение времени суток по введенному часу.

Алгоритм работы:
\begin{vvsu_list}
  \item Пользователь вводит час (целое число от 0 до 23).
  \item Программа проверяет час по диапазонам и выводит соответствующее время суток.
  \item Если введенное число больше 23, выводится ошибка.
\end{vvsu_list}

\begin{vvsu_figure}{Листинг программы для задания 9}{fig:9}
  \begin{minipage}{.75\textwidth}
    \lstinputlisting[language=Python,basicstyle=\fontsize{10}{10}\linespread{1}\selectfont\ttfamily]{9.py}
  \end{minipage}
\end{vvsu_figure}

Ключевые элементы кода:
\begin{vvsu_list}
  \item int(input()) - получение и преобразование часа.
  \item Ряд elif: Если введенное число = 0-4 часа (hour < 5), то print('Сейчас утро'), Если введенное число = 5-11 часов (4 < hour < 12), то print('Сейчас день'), Если введенное число = 12-17 часов (11 < hour < 18), то print('Сейчас вечер'), Если введенное число = 12-17 часов (11 < hour < 18), то print('Сейчас вечер'), Если введенное число = 18-23 часа (17 < hour < 24), то print('Сейчас ночь').
  \item Если введенное число > 23 часов, то print('Ошибка').
\end{vvsu_list}

% Подглава - Задание 10
\subsection{Задание 10}

Задача данной программы это проверка, является ли введенное число простым или составным.

Алгоритм работы:
\begin{vvsu_list}
  \item Код запрашивает у пользователя ввод числа и проверяет его. 
  \item Он проверяет выполнение трех условий в цикле, пока не будет получено действительное число: Если первый символ является буквой, он выводит сообщение об ошибке. Если число меньше 1, он выводит сообщение об ошибке. Если число положительное (больше 0) и число, умноженное на 1, равно числу (что всегда верно для чисел), то оно разбивается на части. 
  \item Затем подсчитывается количество делителей числа. Если число больше или равно 1 и имеет не более 2 делителей (1 и само по себе), оно является простым; в противном случае оно является составным.
\end{vvsu_list}

\begin{vvsu_figure}{Листинг программы для задания 10}{fig:10}
  \begin{minipage}{.75\textwidth}
    \lstinputlisting[language=Python,basicstyle=\fontsize{10}{10}\linespread{1}\selectfont\ttfamily]{10.py}
  \end{minipage}
\end{vvsu_figure}

Пояснение работы программы:
\begin{vvsu_list}
  \item while: - цикл cначала проверяет является ли “num” числом, если да то выводит 'Ошибка: input не является числом', потом проверяет является ли значение “num” меньше 1, если да то выводит 'Ошибка: input меньше 1', если num – чисто и больше 1 то цикл останавливается.
  \item цикл for: - Считает количество делителей.
  \item if: - Если делителей меньше двух или два, выводит что это простое число, иначе выводит что это составное число).
\end{vvsu_list}

\end{document}
