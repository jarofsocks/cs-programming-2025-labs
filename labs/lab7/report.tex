\documentclass[]{vvsu}
\usepackage{csquotes}

\vvsuyear{2025}

%%%%%%%%%%%%%%%%%%%

\usepackage{graphicx} % для изображений
\usepackage{tabularray} % для таблиц
\usepackage{siunitx} % для обозначений (процент, градус)
\usepackage{listings} % для листингов кода

% Список путей, где будут искаться изображения и файлы
\graphicspath{{images/}}

% Файл со списком источников (не используется)
% \addbibresource{./references.bib}

% Автор документа
\author{С.Е. Казюта}

% Настройка стилей для листингов кода
\input{listing_styles.tex}

%%%%%%%%%%%%%%%%%%%

\begin{document}

% Шапка
\vvsuhead{\linespread{1}\selectfont{}МИНОБРНАУКИ РОССИИ\\
\vspace{10pt}Федеральное государственное бюджетное образовательное учреждение\\
высшего образования\\
\fontsize{13}{13}\selectfont{}<<ВЛАДИВОСТОКСКИЙ ГОСУДАРСТВЕННЫЙ УНИВЕРСИТЕТ>>\\
(ФГБОУ ВО <<ВВГУ>>)\\
\vspace{10pt}\fontsize{12}{12}\selectfont{}ИНСТИТУТ ИНФОРМАЦИОННЫХ ТЕХНОЛОГИЙ И АНАЛИЗА ДАННЫХ\\
КАФЕДРА ИНФОРМАЦИОННЫХ ТЕХНОЛОГИЙ И СИСТЕМ}

% Название отчета
\title{Отчет\\по лабораторной работе №7}
\subtitle{по дисциплине\\<<Информатика и программирование>>}

% Участники работы
\member{Студент\\ гр. БИН-25-2}{С.Е. Казюта}
\member{Ассистент\\ преподавателя}{М.В. Водяницкий}

% Вывод титульника
\maketitle

% Задание
% Задание
\begin{addition}{Задание}
  Выполнить задания на Python и оформить отчет по стандартам ВВГУ.

  \textit{\textbf{Задание 1.}}  

Имеется список объектов Фонда с указанием уровня угрозы:

```python
objects = [
    ("Containment Cell A", 4),
    ("Archive Vault", 1),
    ("Bio Lab Sector", 3),
    ("Observation Wing", 2)
]
```

Используя `sorted` и лямбда-выражение, отсортируйте объекты по возрастанию уровня угрозы

  \textit{\textbf{Задание 2.}}  

Дан список сотрудников Фонда с количеством проведенных смен и стоимостью одной смены:

```python
staff-shifts = [
    ("name": "Dr. Shaw", "shift-cost": 120, "shifts": 15),
    ("name": "Agent Torres", "shift-cost": 90, "shifts": 22),
    ("name": "Researcher Hall", "shift-cost": 150, "shifts": 10)
]
```

Используя `map` и лямбда-выражение, создайте список общей стоимости работы каждого сотрудника

Затем найдите максимальную стоимость с помощью `max`

  \textit{\textbf{Задание 3.}}  

Дан список персонала с уровнем допуска:

```python
personnel = [
    ("name": "Dr. Klein", "clearance": 2),
    ("name": "Agent Brooks", "clearance": 4),
    ("name": "Technician Reed", "clearance": 1)
]
```

Используя `map` и лямбда-выражение, создайте новый список, где каждому сотруднику добавляется категория допуска:

* `"Restricted"` - уровень 1
* `"Confidential"` - уровни 2–3
* `"Top Secret"` - уровень 4 и выше

Результат должен быть списком словарей

  \textit{\textbf{Задание 4.}}  

Дан список зон Фонда с указанием времени активности (в часах):

```python
zones = [
    ("zone": "Sector-12", "active-from": 8, "active-to": 18),
    ("zone": "Deep Storage", "active-from": 0, "active-to": 24),
    ("zone": "Research Wing", "active-from": 9, "active-to": 17)
]
```

Используя `filter` и лямбда-выражение, выберите зоны, которые полностью работают в дневной период (с 8 до 18 включительно)

  \textit{\textbf{Задание 5.}}  

Фонд анализирует служебные отчеты. Некоторые отчеты содержат внешние ссылки, которые должны быть удалены перед архивированием

```python
reports = [
    ("author": "Dr. Moss", "text": "Analysis completed. Reference: http://external-archive.net"),
    ("author": "Agent Lee", "text": "Incident resolved without escalation."),
    ("author": "Dr. Patel", "text": "Supplementary data available at https://secure-research.org"),
    ("author": "Supervisor Kane", "text": "No anomalies detected during inspection."),
    ("author": "Researcher Bloom", "text": "Extended observations uploaded to http://research-notes.lab"),
    ("author": "Agent Novak", "text": "Perimeter secured. No external interference observed."),
    ("author": "Dr. Hargreeve", "text": "Full containment log stored at https://internal-db.scp"),
    ("author": "Technician Moore", "text": "Routine maintenance completed successfully."),
    ("author": "Dr. Alvarez", "text": "Cross-reference materials: http://crosslink.foundation"),
    ("author": "Security Officer Tan", "text": "Shift completed without incidents."),
    ("author": "Analyst Wright", "text": "Statistical model published at https://analysis-hub.org"),
    ("author": "Dr. Kowalski", "text": "Behavioral deviations documented internally."),
    ("author": "Agent Fischer", "text": "Additional footage archived: http://video-storage.sec"),
    ("author": "Senior Researcher Hall", "text": "All test results verified and approved."),
    ("author": "Operations Lead Grant", "text": "Emergency protocol draft shared via https://ops-share.scp")
]
```

Используя `filter` и лямбда-выражение:

1. Отберите отчеты, содержащие ссылки (`http` или `https`)
2. Преобразуйте их так, чтобы вместо ссылки отображалось `[ДАННЫЕ УДАЛЕНЫ]`

  \textit{\textbf{Задание 6.}}  

Дан список SCP-объектов с указанием их класса содержания:

```python
scp-objects = [
    ("scp": "SCP-096", "class": "Euclid"),
    ("scp": "SCP-173", "class": "Euclid"),
    ("scp": "SCP-055", "class": "Keter"),
    ("scp": "SCP-999", "class": "Safe"),
    ("scp": "SCP-3001", "class": "Keter")
]
```

Используя `filter` и лямбда-выражение, сформируйте список SCP-объектов, которые требуют усиленных мер содержания

> ⚠️ К объектам с усиленными мерами относятся все SCP, **класс которых не равен `"Safe"`**

Результат должен быть списком словарей исходного формата

  \textit{\textbf{Задание 7.}}  

Дан список инцидентов с количеством задействованного персонала:

```python
incidents = [
    ("id": 101, "staff": 4),
    ("id": 102, "staff": 12),
    ("id": 103, "staff": 7),
    ("id": 104, "staff": 20)
]
```

Используя `sorted` и лямбда-выражение:

1. Отсортируйте инциденты по количеству персонала
2. Оставьте только три наиболее ресурсоемких инцидента

  \textit{\textbf{Задание 8.}}  

Дан список протоколов безопасности и их уровней критичности:

```python
protocols = [
    ("Lockdown", 5),
    ("Evacuation", 4),
    ("Data Wipe", 3),
    ("Routine Scan", 1)
]
```

Используя `map` и лямбда-выражение, создайте новый список строк вида:

```text
"Protocol Lockdown - Criticality 5"
```

  \textit{\textbf{Задание 9.}}  

Имеется список смен охраны с указанием длительности (в часах):

```python
shifts = [6, 12, 8, 24, 10, 4]
```

Используя `filter` и лямбда-выражение, выберите только те смены, которые:

* длятся не менее 8 часов
* не превышают 12 часов

  \textit{\textbf{Задание 10.}}  

Дан список сотрудников с результатами психологической оценки (от 0 до 100):

```python
evaluations = [
    ("name": "Agent Cole", "score": 78),
    ("name": "Dr. Weiss", "score": 92),
    ("name": "Technician Moore", "score": 61),
    ("name": "Researcher Lin", "score": 88)
]
```

Используя `max` и лямбда-выражение, определите сотрудника с наивысшей оценкой

Результатом должно быть имя сотрудника и его балл

\end{addition}

% Содержание
\toc

% Глава - Выполнение работы
\section{Выполнение работы}

% Подглава - Задание 1
\subsection{Задание 1}

Задача данной программы это отсортировать объекты по возрастанию уровня угрозы.

Алгоритм работы:
\begin{vvsu_list}
  \item Программа сортирует объекты по возрастанию уровня угрозы
  \item Программа выводит список объектов.
\end{vvsu_list}

\begin{vvsu_figure}{Листинг программы для задания 1}{fig:1}
  \begin{minipage}{.75\textwidth}
    \lstinputlisting[language=Python,basicstyle=\fontsize{10}{10}\linespread{1}\selectfont\ttfamily]{1.py}
  \end{minipage}
\end{vvsu_figure}

Ключевые элементы кода:
\begin{vvsu_list}
  \item objects = [] - список сортируемых объектов.
  \item print(sorted(objects, key=lambda x: x[1])) - программа выводит сортированный список объектов по возрастанию.
\end{vvsu_list}

% Подглава - Задание 2
\subsection{Задание 2}

Задача данной программы это найти максимальную стоимость работы сотрудника.

Алгоритм работы:
\begin{vvsu_list}
  \item Программа расчитывает общую стоимость каждого сотрудника.
  \item Программа выводит самую большую.
\end{vvsu_list}

\begin{vvsu_figure}{Листинг программы для задания 2}{fig:2}
  \begin{minipage}{.75\textwidth}
    \lstinputlisting[language=Python,basicstyle=\fontsize{10}{10}\linespread{1}\selectfont\ttfamily]{2.py}
  \end{minipage}
\end{vvsu_figure}

Ключевые элементы кода:
\begin{vvsu_list}
  \item staff-shifts = [] - список сотрудников, их зарплат и смен.
  \item output = list(map(lambda a : a['shift-cost'] * a['shifts'], staff-shifts)) - программа создаёт список зарплат умноженных на количество смен сотрудников.
  \item print(max(output)) - программа выводит самое большое число.
\end{vvsu_list}

% Подглава - Задание 3
\subsection{Задание 3}

Задача данной программы это создать новый списокна основе старого, но где каждому сотруднику добавляется категория допуска.

Алгоритм работы:
\begin{vvsu_list}
  \item Программа создаёт новый список из сторого с нужными нам изменениями.
  \item Программа выводит содержимое нового списка.
\end{vvsu_list}

\begin{vvsu_figure}{Листинг программы для задания 3}{fig:3}
  \begin{minipage}{.75\textwidth}
    \lstinputlisting[language=Python,basicstyle=\fontsize{10}{10}\linespread{1}\selectfont\ttfamily]{3.py}
  \end{minipage}
\end{vvsu_figure}

Ключевые элементы кода:
\begin{vvsu_list}
  \item personnel = [] - список сотрудников и их категорий допуста (в числовом виде).
  \item access-levels = () - словарь категорий доступа.
  \item clearence-personnel = map(lambda a : ('name' : a['name'], "clearance" : access-levels[f"(a["clearance"])"]), personnel) - программа создаёт новый список из словарей с категориями доступа переведёнными из чисел в слова.
  \item for i in clearence-personnel:   print(i) - выводит значения списка
\end{vvsu_list}

% Подглава - Задание 4
\subsection{Задание 4}

Задача данной программы это найти зоны которые полностью работают в дневной период (с 8 до 18 включительно).

Алгоритм работы:
\begin{vvsu_list}
  \item Программа находит нужные нам зоны.
  \item Программа выводит нужные нам зоны.
\end{vvsu_list}

\begin{vvsu_figure}{Листинг программы для задания 4}{fig:4}
  \begin{minipage}{.75\textwidth}
    \lstinputlisting[language=Python,basicstyle=\fontsize{10}{10}\linespread{1}\selectfont\ttfamily]{4.py}
  \end{minipage}
\end{vvsu_figure}

Ключевые элементы кода:
\begin{vvsu_list}
  \item zones = [] - список зон.
  \item day-zones = list(filter(lambda x: x["active-from"] == 8 and x["active-to"] == 18, zones)) - программа находит все временные зоны работающие с 8 до 18. 
  \item print(day-zones) - программа выводит найденные ранее временные зоны.
\end{vvsu_list}

% Подглава - Задание 5
\subsection{Задание 5}

Задача данной программы это заменить все ссылки в отчётах на `[ДАННЫЕ УДАЛЕНЫ]`.

Алгоритм работы:
\begin{vvsu_list}
  \item Программа находит все отчёты с ссылками.
  \item Программа заменяет ссылки в отчёиах с ссылками на `[ДАННЫЕ УДАЛЕНЫ]`.
  \item Программа выводит список отредактированных отчётов.
\end{vvsu_list}

\begin{vvsu_figure}{Листинг программы для задания 5}{fig:5}
  \begin{minipage}{.75\textwidth}
    \lstinputlisting[language=Python,basicstyle=\fontsize{10}{10}\linespread{1}\selectfont\ttfamily]{5.py}
  \end{minipage}
\end{vvsu_figure}

Ключевые элементы кода:
\begin{vvsu_list}
  \item reports = [] - список словарей с отчётами.
  \item bad-reports = list(filter(lambda x: 'http' in x["text"], reports)) - программа находит все отчёты с ссылками.
  \item for i in bad-reports:   i['text'] = f'(i['text'][:i['text'].index('http')])('[ДАННЫЕ УДАЛЕНЫ]')' - программа заменяет отчёт на такой-же но с заменённой на '[ДАННЫЕ УДАЛЕНЫ]' ссылкой.
  \item print(i) - программа выводит словарь с изменённым отчётом.
\end{vvsu_list}

% Подглава - Задание 6
\subsection{Задание 6}

Задача данной программы это сформирование списка SCP-объектов, которые требуют усиленных мер содержания.

Алгоритм работы:
\begin{vvsu_list}
  \item Программа фильтрует объекты по классам и выбирает всех кроме безобидных.
  \item Программа формирует список объектов из оставшихся.
\end{vvsu_list}

\begin{vvsu_figure}{Листинг программы для задания 6}{fig:6}
  \begin{minipage}{.75\textwidth}
    \lstinputlisting[language=Python,basicstyle=\fontsize{10}{10}\linespread{1}\selectfont\ttfamily]{6.py}
  \end{minipage}
\end{vvsu_figure}

Ключевые элементы кода:
\begin{vvsu_list}
  \item scp-objects = [] - список объектов.
  \item baddies = list(filter(lambda x: x["class"] != "Safe", scp-objects)) - программа убирает из списка все объекты с классов 'Safe'.
  \item for i in baddies:  print(i) - программа выводит список оставшихся объектов.
\end{vvsu_list}

% Подглава - Задание 7
\subsection{Задание 7}

Задача данной программы это нахождение трёх наиболее ресурсоемких инцидентов.

Алгоритм работы:
\begin{vvsu_list}
  \item Программа сортирует инцидерны.
  \item Программа выводит первые три из отсортированных.
\end{vvsu_list}

\begin{vvsu_figure}{Листинг программы для задания 7}{fig:7}
  \begin{minipage}{.75\textwidth}
    \lstinputlisting[language=Python,basicstyle=\fontsize{10}{10}\linespread{1}\selectfont\ttfamily]{7.py}
  \end{minipage}
\end{vvsu_figure}

Ключевые элементы кода:
\begin{vvsu_list}
  \item incidents = [] - список инцидентов.
  \item new-incidents = sorted(incidents, key=lambda x: x['staff'], reverse=True) - программа сортирует инциденты по количеству ушедших в мир иной в порядке уменьшения.
  \item for i in range(3):   print(new-incidents[i]) - программа выводит первые три объекта из списка.
\end{vvsu_list}

% Подглава - Задание 8
\subsection{Задание 8}

Задача данной программы это создание нового списка f-строк на основе старого списка.

Алгоритм работы:
\begin{vvsu_list}
  \item Программа создаёт список f-строк используя информацию из старого списка.
  \item Программа выводит новый список.
\end{vvsu_list}

\begin{vvsu_figure}{Листинг программы для задания 8}{fig:8}
  \begin{minipage}{.75\textwidth}
    \lstinputlisting[language=Python,basicstyle=\fontsize{10}{10}\linespread{1}\selectfont\ttfamily]{8.py}
  \end{minipage}
\end{vvsu_figure}

Ключевые элементы кода:
\begin{vvsu_list}
  \item protocols = [] - список протоколов.
  \item output = list(map(lambda x : f'Protocol (x[0]) - Criticality (x[1])', protocols)) - программа создаёт список f-строк используя информацию из старого списка.
  \item for i in output:   print(i) - программа выводит новый список.
\end{vvsu_list}

% Подглава - Задание 9
\subsection{Задание 9}

Задача данной программы это найти только те смены, которые: длятся не менее 8 часов, не превышают 12 часов.

Алгоритм работы:
\begin{vvsu_list}
  \item Программа отфильтровывает среди всех смен только те, которые длятся не менее 8 часов и не превышают 12 часов.
  \item Программа выводит отфильтрованный список смен.
\end{vvsu_list}

\begin{vvsu_figure}{Листинг программы для задания 9}{fig:9}
  \begin{minipage}{.75\textwidth}
    \lstinputlisting[language=Python,basicstyle=\fontsize{10}{10}\linespread{1}\selectfont\ttfamily]{9.py}
  \end{minipage}
\end{vvsu_figure}

Ключевые элементы кода:
\begin{vvsu_list}
  \item shifts = [] - список длительности смен.
  \item some-shifts = list(filter(lambda x : x >= 8 and x <= 12, shifts)) - программа фильтрует список длительностей смен из которых выберает только те, которые длятся не менее 8 часов и не превышают 12 часов.
  \item for i in some-shifts:   print(i) - пргрмма выводит список смен подходящих по нашим критериям.
\end{vvsu_list}

% Подглава - Задание 10
\subsection{Задание 10}

Задача данной программы это имя и балл сотрудника с наивысшим баллом.

Алгоритм работы:
\begin{vvsu_list}
  \item Программа находит сотрудника с нужной нам отличительной чертой.
  \item Программа выводит данные сотрудника.
\end{vvsu_list}

\begin{vvsu_figure}{Листинг программы для задания 10}{fig:10}
  \begin{minipage}{.75\textwidth}
    \lstinputlisting[language=Python,basicstyle=\fontsize{10}{10}\linespread{1}\selectfont\ttfamily]{10.py}
  \end{minipage}
\end{vvsu_figure}

Пояснение работы программы:
\begin{vvsu_list}
  \item evaluations = [] -список словарей синформацией об сотрудниках.
  \item output = max(evaluations, key=lambda x : x['score']) - программа находит сотрудника с наивысшим баллом.
  \item print(f'(output['name']): (output['score'])') - программа выводит f-строкой имя и балл сотрудника с наивысшим баллом.
\end{vvsu_list}

\end{document}
